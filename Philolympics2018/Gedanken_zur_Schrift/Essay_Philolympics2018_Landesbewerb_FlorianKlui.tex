\documentclass[12pt,a4paper,oneside]{article}
\usepackage[utf8]{inputenc}
\usepackage[german]{babel}
\usepackage[T1]{fontenc}
\usepackage{amsmath}
\usepackage{amsfonts}
\usepackage{amssymb}
\usepackage{graphicx}
\usepackage{endiagram}
\usepackage{enumitem}

\author{Florian Kluibenschedl}
\title{Essay: Über die Bedeutung der Schrift}

\DeclareSIUnit{\calory}{cal}

\begin{document}
  \maketitle
  
  \textit{\glqq Nur noch Historiker und andere Spezialisten werden in Zukunft Schreiben und Lesen lernen müssen.\grqq} - Villem Flusser: Die Schrift. Hat Schreiben Zukunft? 1987 \\
  
  Vor dem Hintergrund der digitalen Revolution scheint es, als ob die Schrift zu verschwinden scheint!? Der Mensch ist mehr und mehr gebunden an die digitale Welt. Immer mehr werden Dinge ausgelagert und den Computern dieser Welt überlassen. So rechnet kein Mensch mehr eine Simulation des Wetters per Hand oder spielt Schach mit der Absicht des Gewinnens gegen einen Computer.\\
  
  Vor allem die Entwicklungen der letzten fünf Jahre regen demnach zum Denken an. Die Meldungen lauten in etwa wie folgt: Facebook-Ingenieure verstehen die selbst-programmierten Sprachroboter nicht mehr, Drohnen übernehmen Kriegsentscheidungen, usw. Was dies mit Schrift an sich zu tun hat? - Recht viel. Der Mensch ist \glqq Schöpfer\grqq dieser Möglichkeiten und bediente sich bei deren Erschaffung der Schrift und Sprache. Die Frage, die sich nun stellt ist, ob diese Entwicklungen den Grund ihrer Existenz selbst entmündigen können?
  
  Betrachten wir das Zitat von Flusser unter einem zeitgeschichtlichn Gesichtspunkt: im Jahre 1987 war die technische Entwicklung noch nicht so fortgeschritten wie heute - an den standardmäßigen Besitz eines Computers war nicht im Traum zu denken. Trotzdem wurde der Verlust der Sprache befürchtet. Heute, 30 Jahre später möchte ich behaupten, dass trotz der immer rasanteren Entwicklung der Technik - der angenommenen Ursache für das Verschwinden der Schrift - die Schrift nichts an Bedeutung eingebüßt hat.
  
  Hat man sich dazumals gänzlich geirrt oder hat man sich nur etwas in der Zeit und den damit verbundenen größeren oder geringeren Möglichkeiten vertan? Dies sollte im Folgenden näher analysiert werden. \\
  
  Um diese Frage erst einmal besser verstehen zu können, machen wir einen Exkurs und betrachten auf diesem Weg die Sprache an sich und wie man sie sich in all ihren Ausprägungen anschaulich verständlich machen kann.
  
  In meinen Betrachtungen tätige ich die Annahme, dass Sprache mit einer der Gründe ist, die den Menschen zu dem machen, was er heute ist. So hilft sie, den Menschen vom Affen abzugrenzen und ihn zum \glqq Herrscher der Welt\grqq zu machen. Doch was ist Sprache? Lässt sie sich auf einen gemeinsamen Nenner reduzieren und somit begreifbar machen, wo sie doch so vielfältig ist? Versuchen wir es mit folgender Überlegung:
  
  Fassen wir Sprache als einen Quader auf mit einer Breit, Höhe und Tiefe. Jede dieser drei Seiten repräsentiert eine Erscheinungsform der Sprache. So lege ich willkürlich die Breite als Ausdrucksform in Kunst\footnote{Kunst wird deswegen als Ausdrucksform der Sprache angenommen, da sie über das mit Schrift darstellbare hinausgeht und Dinge auf eine Art und Weise darstellt, die unkonventionell sind und es demnach ermöglicht, die Methode der Erkenntnisgewinnung gemäß einem Ziel der Philosophie des Findens von Definitionen erweitert}, die Höhe als mündliche Form und Tiefe als schriftliche Erscheinung fest. Die Einheiten dieser drei Seiten stellen jeweils die relativen Möglichkeiten der jeweiligen Form dar. Das Volumen des Quaders repräsentiert die sprachlichen Möglichkeiten und folgt einem direkt proportionalen Zusammenhang. Es gilt, je größer das Volumen, desto größer die sprachlichen Möglichkeiten. \\
  
  Diese Betrachtung eignet sich nun einerseits zum Darstellen der sprachlichen Möglichkeiten einer Einzelperson, was an folgendem Beispiel illustriert werden soll: nehmen wir als Person einen Künstler, der sowohl sprachlich als auch musikalisch\footnote{also künstlerisch} aktiv ist. Er besitzt demnach relativ hohe sprachliche Möglichkeiten, da bei ihm zwei der drei Seiten des Quaders stark ausgeprägt sind. Im Vergleich zu einem Schriftsteller, bei dem besonders die schriftliche Ebene sehr groß ist, hat der Künstler größere sprachliche Möglichkeiten\footnote{natürlich nur unter der Annahme, dass die Produkte des Schriftstellers nicht als Kunst zu erachten sind}.
  
  Andererseits kann diese Anschauungsweise genützt werden, um die sprachlichen Möglichkeiten der Menschheit zu analysieren. Bei dieser Betrachtungsweise fällt auf, dass die Größe des Quaders mit dem Entwicklungszustand der Menschheit mehr oder weniger stark korreliert. So ermöglicht erst die Erweiterung der Schrift um die Schreibweisen der Mathematik die Entwicklung von komplexen Computersystemen\footnote{die es wiederum ermöglichen, dass dieser Text in einer für jeden leserlichen Form verfasst wird}. In diesem Fall wurde das relative Volumen des Quaders vergrößert und ermöglichte Entwicklung.\\
  
  Es liegt somit nahe, anzunehmen, dass eine Entwicklung des Menschen einhergeht mit dessen sprachlichen Möglichkeiten - sprich, der Größe des Volumens des Quaders. Aus dieser nüchternen mathematischen Sichtweise betrachten wir die Möglichkeiten, die sich ergeben für eine angestrebte Vergrößerung des Quaders. Im idealsten Fall werden alle drei Seiten vergrößert. Es ist aber genauso möglich, nur eine oder zwei Seiten zu vergrößern. Wie sieht die Situation im Falle des Zitates nun aus? Es werden die Möglichkeiten der Schrift verkleinert, was grundsätzlich jedoch nicht heißt, dass deswegen das Volumen kleiner wird, da ja die anderen beiden Seiten genauso größer werden können. Es stellt sich nur die Frage, ob ein solches Szenario als realistisch zu erachten sei? Man bedenke außerdem, was ein Verlust von Schrift bedeuten würde - eine Seite würde auf die Größe null zusammenschrumpfen, was für das Volumen und die damit verbundenen sprachlichen Möglichkeiten denselben Fall bedeuten würde\footnote{ein solches Szenario ist für den Autor auch zweifelhaft, er möchte deswegen diese Annahme lediglich im Raum stehen lassen und sieht diese Konsequenz sich nur aus den mathematischen Möglichkeiten ergebend} Doch macht dies Sinn?
  
  Betrachten wir dazu den Zeck und die Möglichkeiten der Schrift etwas genauer:
  
  Schrift dient zum Einen zur Dokumentation. Dies ist vermutlich einer der ursprünglichsten Gründe, warum sie überhaupt entstand. Man benötigte sie, um Dinge festzuhalten, an Generationen weiterzugeben\footnote{was mit einem notgedrungenem Wissenstransfer - der für die Entwicklung und Fortschritt notwendig ist - einhergeht}. Ebenso ist sie ein Mittel, um die Wirklichkeiten in denen die einzelnen Menschen leben von einem Individuum zum anderen zu transferieren\footnote{für den Physiker bietet sich ein Vergleich mit der Koordinatentransformation an - Sprache und Schrift sind \textit{Koordinatentransformationen} zwischen den Wirklichkeiten der Menschen}. Der letzte Punkt, den ich nun aufzähle behandelt die Unzulänglichkeit unseres Gehirns. Dieses ist zu erstaunlichen Leistungen fähig, doch mangelt es meines Erachtens am sogenannten Arbeitsspeicher, der entweder zu knapp bemessen oder zu sehr von den für uns als selbstverständlich erachteten die Dingen in Anspruch genommen wird\footnote{Bilderkennung, unbewusste Denkprozesse, unbewusste Steuerung der Muskulatur, usw.}. Den Grund außer Acht lassend, denke ich, dass der Arbeitsspeicher zu knapp bemessen ist. Die wenigsten können sich eine Nummer auf den ersten Blick merken oder Schlussfolgerungen, die drei oder mehr aufeinanderfolgende Schritte inkludieren beim ersten Mal im Gedächtnis behalten. Aufgrund dieser Unzulänglichkeit hat der Mensch die Schrift kreiert. Mit ihr in allen Erscheinungsformen kann er all die Dinge darstellen, die er sich nur schwer oder gar nicht vorstellen kann. Er bringt seine Ideen zu Papier und benützt das zu Papier Gebrachte, um neue Ideen zu bekommen - ein schöner Kreisprozess.\\
  
  Bevor ich ein Urteil darüber abgebge, ob Schrift nun Zukunft hat oder nicht, möchte ich die bisherigen Erkenntnisse zusammenfassen und anschließend eroieren, was man als Ersatz für Schrift verwenden könnte:
  
  Die sprachlichen Möglichkeiten können als Quader mit drei Seiten alias Eigenschaften aufgefasst werden, dessen Volumen mit den soeben genannten korreliert. Eine Vergrößerung des Volumens ist erstrebenswert, wenn das Ziel die Weiterentwicklung der Menschheit sein sollte. Schrift als eine Seite des Quaders erfüllt mehrere Aufgaben, unter anderem die der Kompensation der Unzulänglichkeit des Menschen. Schrift dient demnach zum Festhalten von Ideen und hilft anderen Menschen, diese Ideen in ihre Wirklichkeiten\footnote{der Begriff der Wirklichkeiten wird in Anlehnung an jene von Paul Watzlawick getätigt} aufzunehmen. Welche Möglichkeiten existieren nun, dies zu kompensieren wenn Schrift nicht mehr vorhanden ist, wie von Flusser prognostiziert? Ist es überhaupt wert, dies zu kompensieren oder gibt es eindeutig bessere Möglichkeiten?
  
  Mit diesen Fragen stellen sich notgedrungen weitere. So stellt sich die Frage nach dem Sinn und Ziel des Lebens, ob eine ständige Weiterentwicklung des Menschen nötig sei oder ob er auch in einem statischem Zustand verweilen könnte/sollte. Unter Berücksichtigung der Bedürfnispyramide kann man sich fragen, ob diese einem nicht irgendwie durch biologische und neuropsychologische Methoden einem vorgegaukelt werden kann. Ausgehend von der Fage, ob der Mensch die Schrift überhaupt braucht, kommt man wieder bei der Sinnfrage an. Man sieht, die Fragen schließen einen Kreis und führen von einer zu anderen. Eine sinnvolle Betrachtung und Lösung scheint nicht möglich.\\
  
  Ich versuche, die Sache aus evolutionärer und technischer Sichtweise zu beantworten. Sich die ursprüngliche Entwicklung der Schrift betrachtend, erkennt man, dass sie aus der Notwendigkeit heraus entstand, wie bereits zuvor erwähnt. Sie entwickelte sich nach und nach, was bedeutet, dass sie immer wieder adaptiert wurde. Es wäre deswegen nichts darüber einzuwenden, dass sie einen weiteren Entwicklungsschritt tätigt und einfach verschwindet. Was ihren Platz einnimmt bzw. einnehmen könnte, kann aus einer technisch-, biologischen Sichtweise heraus betrachtet werden. Wenn Schrift die beschriebene Unzulänglichkeit des Gehirns kompensiert, dann gibt es eine Möglichkeit, das Gehirn selbst so zu erweitern, dass es der Kompensation durch die Schrift nicht mehr bedarf?
  
  Die letzten Entwicklungen auf den Gebieten der Biologie, Neurologie und Technik haben viel Potential. Mithilfe von MRT Geräten können bereits Bilder von Gedanken des Gehirns gemacht weden. Denkt man beispielsweise an einen Hund, kann dieses im Gehirn entstehende Bild auf einem Bildschirm dargestellt weden\footnote{die Auflösung lässt zwar noch zu wünschen übrig, man ist jedoch im Begriff, diese zu vergrößern}. Portable EEG Geräte messen Gehirnströme und werden bereits in vielen Forschungsprojekten verwendet, um Personen im alltäglichen Leben zu beobachten und damit bessere Einblicke in die Denkstrukturen des Gehirns zu erlangen. Vor einigen Jahren ist die erste Kommunikation nur über Gedanken zwischen zwei Personen gemacht worden. Ausgehend von diesen Beispielen möchte ich behaupten, dass die Forschung drauf und dran ist, die Unzulänglichkeit des Gehirns zu kompensieren\footnote{den Zeitpunkt der ersten Resultate ihrer Forschung kann natürlich niemand abschätzen}. Damit wäre die Frage des Möglichen beantwortet - nun zu jener des sinnvollen, die sich für mich als weitaus komplizierter darstellt.
  
  Betrachten wir hierzu, wie das Leben funktioniert und versuchen daraus, einen Sinn abzuleiten. Eine treibende Kraft der Natur, wie sie direkt unser Leben beeinflusst ist das Prinzip des Stärkeren, auf dem Darwin seine Evolutionstheorie begründet. Er meint damit, dass jener überlebt, der die Möglichkeiten besser ausnützt und demnach überlebt. Alles andere wird, durch was auch immer, ausgeschieden. Wenn nun das Prinzip des Besseren/Stärkeren gilt, könnte dies nicht bedeuten, dass das Finden des Besten/Stärksten das Ziel ist? Es ist wie mit dem Finden eines Minima einer Funktion, deren es bei komplexen Funktionen viele gibt. Das Ziel des Lebens ist das Finden eines globalen Minima, auf dem man sich letztlich ausruhen kann. Der Trick dabei? Es gibt kein globales Minima, sondern nur lokale, die an sich unbefriedigend sind\footnote{ein anschaulicher Vergleich bietet sich mit dem Prinzip des Ideals von Platon an - es gibt ein ideales Pferd, das man zwar nicht beschreiben kann, dem man sich aber beliebig genau annähern kann. Ich verwende die Überlegung von Minima deswegen, da ich es aus der Mathematik herausnehme, was im weiteren Sinne eine mathematische Deutung und Interpretation des Lebens ermöglicht - es erwächst somit, sofern man das Leben in abstrakter Form als eine Funktion mit Minima und Maxima darstellt eine mathematische Repräsentation, die aufzustellen hier nicht zweckbringend und vor allem in der verfügbaren Zeit nicht möglich ist}. Aus dieser Argumentation ergibt sich, dass der Mensch einer Entwicklung benötigt und sich nicht einfach ausruhen kann. Er kann nicht das Denken an sich als eine Eigenschaft, die ihn zu dem macht, was er ist, einfach so aufgeben und dennoch das bleiben, was er ist.\\
  
  Um zum Ende zu kommen fassen wir alles noch einmal zusammen: es galt, die Behauptung von Villem Flusser, dass Schrift in Zukunft nur noch von Einzelnen beherrscht werden müsse, genauer zu beleuchten. Dazu haben wir uns die Frage gestellt, was Sprache ist und welche Rolle Schrift im Verständnis der Sprache spielt. So haben wir die Betrachtungsweise des Quaders mit den drei Ausprägungen der Sprache in Form von Kunst, Mündlichem und Schrift eingeführt. Sodann haben wir die Rolle der Schrift an sich eingehender erläutert und hergeleitet, dass sie eine Möglichkeit ist, die Unzulänglichkeit des Gehirns auszugleichen. Abschließend stellten wir die Frage, ob die Rolle der Schrift ersetzt werden kann, durch was und ob dies grundsätzlich Sinn mache.
  
  Unter der Annahme, dass die sprachlichen Möglichkeiten, deren Existenz als absolut notwendig erachtet wird, durch Technik tatsächlich verbessert werden, macht es demnach Sinn, Schrift durch diese neuen Möglichkeiten zu ersetzen. Es ist aus evolutionärer Sicht ein Notwendikum, das sich aus dem Suchen und dementsprechendem Finden von Minima/Idealen ergibt. Wenn nun die Schrift als solches ersetzt wird, muss das zu Beginn entwickelte Quadermodell der Sprache erweitert werden.
  
  Ich schlage vor, dass man die Schrift durch Darstellungsart von Gedanken repräsentiert. Dies ist im Moment ein Platzhalter und spiegelt meines Erachtens die Grundintention der Schrift wieder.
  
  Demnach stimme ich Flusser zu und denke, dass das Erlernen der Schrift der Schrift in Zukunft nicht mehr notwendig sein wird, da wir andere Wege gefunden haben werden, uns auszudrücken, die besser sind und es deswegen notwendig ist, diese auch umzusetzen. Wie diese neuen Wege aussehen, entzieht sich meinem Vorstellungsvermögen. Ich empfinde demnach das zukünftige Verschwinden von Schrift als durchaus positiv und als eine Möglichkeit der Weiterentwicklung. Um den Kreis zu schließen, möchte ich mit der Frage aufhören, wo und vor allem wie die Philosophieolympiade in Zukunft aussieht, wenn ich diesen Text nicht mehr schreiben kann. Existiert er dann in meinem Kopf und kann einfach über Copy/Paste Gedanken abgegeben werden? Gibt es überhaupt noch Texte oder nur mehr Darstellungsformen, die aus heutiger Sicht komplett neu sind?
  
  Und um die Unzulänglichkeit des Gehirns zu unterstreichen - nach dem Verfassen dieses Textes ist es mir nicht möglich, alle Argumentationen auf einmal im Kopf in Form eines Bildes zu behalten - nur durch nachdenken gelingt es, die einzelnen Fragmente herauszunehmen und langsam, Schritt für Schritt aneinander zu stückeln. Deswegen habe ich ja meine Gedanken mithilfe der Schrift niedergeschrieben und das Niedergeschriebene dazu verwendet, mir durch Rückkoppelung weitere Gedanken zu machen - ein Kreislauf.
  
\end{document}