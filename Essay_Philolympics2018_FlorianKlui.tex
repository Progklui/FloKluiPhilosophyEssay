\documentclass[12pt,a4paper,twoside]{article}
\usepackage[utf8]{inputenc}
\usepackage[german]{babel}
\usepackage[T1]{fontenc}
\usepackage{amsmath}
\usepackage{amsfonts}
\usepackage{amssymb}
\usepackage{graphicx}
\author{Florian Kluibenschedl}
\title{Essay: Der Weg und die Verantwortung des Lebens}

\begin{document}
  \maketitle
  
  Der vorliegende Essay setzt sich mit dem Weg des Lebens und der Rolle des Menschen auseinander. Als inhaltliche Grundlage wurde ein Zitat von Jean-Paul Sartre gewählt. Zu Beginn wird das Zitat analysiert und die korrespondierende Theorie erweitert\footnote{die Erweiterung sei als relativ anzusehen, da sie darauf beruht, dass der Autor die entsprechenden Punkte der Erweiterung aus dem Zitat nicht direkt herauslesen kann. Da die Punkte der Erweiterung naturgemäß ähnlicher Natur sind, kann es jedoch sein, dass der Autor sie als im Zitat nicht enthaltend annimmt, obwohl sie eigentlich vorhanden sein sollte. Dies relativiert wiederum die Erweiterung an sich, weswegen sie als ein fließender Übergang mit der Theorie an sich anzusehen ist}. Anschließend wird auf Basis dieser Erkenntnisse eine allgemeine Theorie entwickelt, die sich mit dem Leben, seinem Weg und dem Menschen befasst und probiert, alles in einen großen Rahmen zu gießen. \\
  
  \textit{\glqq Es gibt keinen vorgezeichneten Weg, der den Menschen zu seiner Rettung führt; er muss sich seinen Weg unablässig neu erfinden. Aber er ist frei, ihn zu erfinden, er ist verantwortlich, ohne Entscheidung, und seine ganze Hoffnung liegt allein in ihm.\grqq} - Sartre im Interview mit Christian Grisoli: \glqq Entretien avec Jean-Paul Sartre\grqq , Paru 13, Dez. 1945, S. 5-10 \\
  
  Die erste Assoziation, die einem beim Lesen dieses Zitates in den Sinn kommt, ist das Prinzip des freien Willens des Menschen. Sartre legt seinen Ansichten zugrunde, dass der Mensch einen freien Willen besitze. Aus dieser Annahme heraus folgert er, dass der Mensch die Fähigkeit besitzt, seinen Lebensweg neu zu erfinden. Außerdem ergibt sich daraus, dass der Mensch die Entscheidungsfähigkeit besitzt, für sein Handeln vollste Verantwortung zu übernehmen, obwohl man ihn nie fragte, ob er es denn so wolle. 
  
  Er bedient sich somit einem komplett logischem Ablauf, indem er das Prinzip des nicht vorherbestimmten und die gründsätzliche Entscheidungsfreiheit annimmt. Alles andere ergibt sich daraus. Oder etwa nicht? \\
  
  Ein Problem gibt es mit dem Ziel, das er nennt. Das Ziel sei die Rettung des Menschen, erreicht durch den Weg. Dass die Rettung nun auf unterschiedlichste Arten zu bewältigen sei, sei einmal als logische Schlussfolgerung angenommen, doch ist die Frage, ob dies Sinn mache?
  
  Angenommen, zwei Menschen unterhalten sich miteinander. Der eine, reicher Aktieninhaber und der andere, Bauarbeiter auf der Baustelle der neuen Luxusvilla des einen. Beide verfolgen sie unterschiedliche Ziele im Leben, befinden sich somit auf anderen Niveaus. Verallgemeinert und vieles vernachlässigend lassen sich zwei Szenarien beschreiben, wie beide miteinander auskommen. 
  
  Im ersten arbeitet der Bauarbeiter fleißig, ist bemüht und teilt sein Wissen mit dem praktisch etwas unerfahrenen Aktieninhaber. Im Gegenzug wird er dafür fair entlohnt. 
  
  Im zweiten Szenarion sieht der Bauarbeiter die Klassenunterschiede der beiden und entwickelt eine Eifersucht, die darin resultiert, dass er unsauber und nur das absolut notwendigste arbeitet. Im Falle, dass dies dem Aktieninhaber auffällt, wird er ihn nicht fair belohnen (wollen) und ihn im schlimmsten Falle des Bauauftrags entziehen.
  
  Die Frage, die sich nun stellt, ist, inwiefern nun eine Rettung erfolgte? Die Rettung ist das Ziel und wenn man in Retroperspektive auf eine bereits geschehene Handlung sieht, wird sie zum Resultat. Da nun offensichtlich zwei Resultate ersichtlich sind, impliziert dies das Vorhandensein von mehreren Rettungen. Sartres theoretischem Konstrukt zufolge ist dies möglich, doch wurde noch immer nicht erörter, ob dies Sinn mache? 
  
  Im Laufe der Weltgeschichte haben sich die Menschen unterschiedliche Theorien dazu angeeignet. Der Lehre der katholischen Kirche zufolge erfolgt eine Rettung des Menschen durch Jesus Christus - eine externe Person, an deren Wahrhaftigkeit ein Konstrukt an Annahmen geheftet ist, das in sich geschlossen Sinn macht und deswegen für eine Beantwortung der Rettungsfrage herangezogen werden kann. \\
  
  Was ich hiermit zeigen wollte, ist, dass das Ziel des Weges, das Sartre nennt, nicht vorherbestimmt ist und keineswegs als absolut zu bezeichnen ist. Es gilt somit zu bedenken, dass es unterschiedliche Rettungen gibt, wobei diese auf keinen Fall in ihrer Richtigkeit gleichzusetzen und damit gleichwertig sind, wie am Beispiel des Bauarbeiters illustriert. 
  
  Eine Sache fehlt jedoch im Zitat bzw. kann ich sie nicht direkt herauslesen, vorausgesetzt, der Autor habe sie versteckt eingebaut. Ich meine den Rückkoppelungseffekt der Rettung zum Weg. Wenn es einen nicht vorgezeichneten Weg zur Rettung gibt und man grundsätzlich frei ist, diesen zu finden, impliziert dies, dass man jeden Moment eine Rettung durchlebt und, sofern man die Fähigkeit des Speicherns von Inhalten aus der Vergangenheit mit einhergehendem Abrufen in der Gegenwart besitzt\footnote{= Fähigkeit des Nachdenkens über Vergangenes - Reflexion} in der Lage ist, die Rettung bzw. das Ziel zu adaptieren und somit konstant neu zu erfinden. \\
  
  Um kurz die bisherigen Erkenntnisse zusammenzufassen: Das Zitat von Sartre basiert auf den Annahmen der Unvorherbestimmtheit und der freien Entscheidungsfähigkeit. Als Ziel des Weges kann die Rettung des Menschen angesehen werden. 
  
  Hinzugefügt wird, dass es unterschiedliche Arten von Rettungen gibt, die nicht gleichwertig sind.\footnote{auf eine Einteilung in gute und böse Rettungen wird verzichtet, da es als nicht zweckdienend für die Erörterung des Zitates erachtet wird, da es einer ausholenderen Erklärung mit einhergehenden Defintionen bedürfe} Außerdem wurde der Rückkoppelungseffekt, der zwischen Rettung und Weg besteht, eingeführt. \\
  
  An dieser Stelle wird versucht, eine allgemeine Theorie über die eben beschriebenen Zusammenhänge zu legen. 
  
  
\end{document}