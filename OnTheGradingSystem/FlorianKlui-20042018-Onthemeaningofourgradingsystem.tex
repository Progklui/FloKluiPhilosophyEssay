\documentclass[12pt,a4paper]{article}
\usepackage[utf8]{inputenc}
\usepackage[english]{babel}
\usepackage{amsmath}
\usepackage{amsfonts}
\usepackage{amssymb}
\usepackage{graphicx}
\author{Florian Kluibenschedl}
\title{On the meaning of (the) grading system(s)}
\begin{document}
  \maketitle
  
  \section{Introduction}
  The goal of every pupil - wittingly or not - is to receive a grade in every subject at the end of a school year. The grade itself is merely a number written down on a single sheet of paper but it has got tremendous meaning - to the pupil himself, his life and future, his development, character, family, parents, mates, etc.  A grade is responsible for success and failure at the same time and thus carries a lot of burden. In the following the mark and its meaning should be examined to get a more decent understanding of the subject. \\
  
  Considering the mathemical aspect the grade in the certificate is representative for a bunch of numbers each standing for some specific performance of the pupil - whatever this may be. The calculated average results in the final grade assigned.
  
  However, what do the numbers themselves represent? - achievements, thoughts, originality of ideas, creativity?! This all are key words each indicating a noble thing. \\
  
  In order to understand the concept of grading a bit deeper one must investigate why the principle emerged in the first place.\\
  
  \section{A bit of historical reasoning}
  Life is there to be lived - starting with a troubling birth and resulting in a (often even more troubling) death. Somehow, society (or more general - the companionship of other people) has imposed rules on humans. Thus, in order to survive, live life and serve society as good as possible it is crucial to learn. This is possible through adaptation, experience and very often trial and error plays an important role in finding out things. Life can be seen as some kind of experiment or game of which you do not know the outcome. Step by step one learns the rules and is therefor able to perform better - in best cases, excell at the game. 
  
  Though this process of learning is not an easy but rather tremendously hard and challenging one. It is not fast at all. To fasten learning humanity has started to hold on to knowledge by writing it down. This enables other humans to quickly understand old concepts and in return getting inspired to create something new. 
  
  Over time humanity has developed a few concepts which can be generalized in some form. This includes - among many others - the concept of mathematics and the correct use of language. 
  
  In this sense, creating new knowledge means using what you already know to build new things upon this knowledge otherwise you happen to reinvent the wheel over and over again. \\
  
  With the increasing complexity of the world forced by human willpower we can conclude that the need of a concept to instill some basic ideas of the already known in the heads of the new generation becomes apparent and thus the meaning and task of school is more obvious.
  
  With time man recognized that the concepts taugth in school are merely understood by some pupils - while these excell in a subject others experience decent troubles in understanding. With different attitudes towards a subject and a lot of different performances in class a classification system seems to be needed. If every young adult is attending the same institution called school before he is entering the wide variety of jobs and employments it is somehow essential to separate them from each other by assigning each a profile representing their qualities somehow. 
  
  Not everyone is a maths prodigy and thus not everyone is destined to become a mathematician. In this sense a classification alias separation system is required. At the moment this system is implemented in an abstract grading system which is assigning a number to the performance of a pupil. Every grade is thought to be representing specific performances and possibilities later on in life. 
  
  For example obtaining A's in every subject enables one to study whatever is desired. Furthermore it increases the chance of getting certain jobs while at the same decreasing the chance of getting others. We see - the system is embedded deeply in our society and our values. Many future decisions are influenced by the interpretation of the grades by some individual. \\
  
  Though - what does it mean, e.g. obtaining all A's? - that the pupils excelled everywhere? - he was the best in a class full of idiots or the best in a class full of prodigies? - he knew more than the teacher or simply more than required? - he made no mistakes in grammar or did make a lot of mistakes in grammar but instead had a bunch of original ideas? ... We see - there are many possibilities what kind of performance a grade can indicate but typically nobody will consider all these when having to judge the meaning of a grade e.g. in a job interview. \\
  
  So far we have developed an explanation pointing out that a grading system which emerged out of the need to classify the performance of a pupil was and is needed to be able to separate them from each other by giving them different possibilities later on in life based on their received grades. Though there seems to be a tremendously huge problem with this intention. The performance of a pupil can be of various kind and it is definitely not easy to measure it since every pupil has got strengths and weaknesses. The question emerges if solely intelligence or solely endeavor should influence the mark. Would the system then be fair?
  
  The problem is a very fundamental one discussing the question how one can assess the ability to possess knowledge and being able to use this knowledge. Is it enough to know what has already been developed or is it more important to have new ideas and not knowing everything of the past? Somehow a middle way is preferable but then again, which things are important to know and when are they important to be known? Is the simplest answer - that solely the knowledge directly imposed by the environment is needed - enough or just there to be making things unnecessarily easy? - All questions highly interesting but I won't cover them here.\\
  
  \section{Globalisation and its influence on grading}
  Globalistation can be seen as one of the biggest forces currently ruling development worldwide - the development of society and the individual. It affects almost every part of life and thus influences very much the way of education all around the world. 
  
  With globalisation it is very easy to travel from one destiny in the world to an other within 24 hours. It is also possible to look for employment everywhere. That is certainly one of the most crucial things! If there are applicants from all over the world for a programming job in New York the company seeking is somehow in the need to be able to compare the qualifications of each applicant. They do not want to miss out on some highly motivated and skilled dude from Simbabwe simply because the education system is different!? 
  
  In this sense standardized testings have been introduced in many states. This are testings where it is obvious what qualities are needed. Thus they are internationally highly comparable which is exactly what the company in New York wants. \\
  
  Now - governments want to be able to compare the education of their students with the education of other countries too. The PISA test has been introduced and assigns every state some number which represents how well their students are performing in comparison to students worldwide. 
  
  The goal of every state is to score best in this examination which is why they look for ways to prepare their students in the best possible manner. This of course influences what is being taught in class and how various grades are assigned. In this sense the grade is fulfilling a well-defined purpose. Whether that is good or not one has to decide on his own.
  
  \section{The attempt of finding a solution and conclusion}
  As outlined above a grade is merely a number representing some kind of performance of a student. It is nothing more than an attempt to be able to compare the knowledge and general performance and is thus just an abstract concept created by mankind. Grades can be seen as a summary of the qualities of a person enabling others to quickly get an overview of the performance, capabilities, knowledge, etc. a particular person possesses without having to meet in person or getting to know him intensely by talking, spending time, etc. One can see the grade as a shortcut for evaluation purposes.
  
  It can be derived that grades are essential in our society though it is always good to bear in mind what each grade stands for when evaluating the meaning of it. This analysis is often quite complex and involves knowing the influence of globalisation, governments, teachers, behaviour of the student himself and many more things. 
  
  As soon as these things are known to the observer it is a pleasure to evaluate a certain grade because one can be sure to completely understand it without having to fear that ones judgement is full of prejudices.  
  
\end{document}