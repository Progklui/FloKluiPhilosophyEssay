\documentclass[12pt,a4paper,oneside]{article}
\usepackage[utf8]{inputenc}
\usepackage[german]{babel}
\usepackage[T1]{fontenc}
\usepackage{amsmath}
\usepackage{amsfonts}
\usepackage{amssymb}
\usepackage{graphicx}
\usepackage{endiagram}
\usepackage{enumitem}

\author{Florian Kluibenschedl}
\title{Essay: Die Minimierungsfunktion des Lebens}

\DeclareSIUnit{\calory}{cal}

\begin{document}
  \maketitle
  
  Der vorliegende Essay setzt sich mit dem Weg des Lebens, seinem Ziel und der Rolle des Menschen in demselben auseinander. Als inhaltliche Grundlage wurde ein Zitat von Jean-Paul Sartre gewählt. Zu Beginn wird das Zitat analysiert und die korrespondierende Theorie erweitert\footnote{die Erweiterung sei als relativ anzusehen, da sie darauf beruht, dass der Autor die entsprechenden Punkte der Erweiterung aus dem Zitat nicht direkt herauslesen kann. Da die Punkte der Erweiterung naturgemäß ähnlicher Natur sind, kann es jedoch sein, dass der Autor sie als im Zitat nicht enthaltend annimmt, obwohl sie eigentlich vorhanden sein sollten. Dies wiederum relativiert die Erweiterung, weswegen sie als ein fließender Übergang mit der Theorie an sich anzusehen ist}. Anschließend wird auf Basis dieser Erkenntnisse eine allgemeine Theorie entwickelt, die sich mit dem Leben, seinem Weg und Ziel sowie mit dem Menschen befasst und probiert, alles in einen großen\footnote{mitunter mathematischen}  Rahmen zu gießen. \\
  
  \textit{\glqq Es gibt keinen vorgezeichneten Weg, der den Menschen zu seiner Rettung führt; er muss sich seinen Weg unablässig neu erfinden. Aber er ist frei, ihn zu erfinden, er ist verantwortlich, ohne Entscheidung, und seine ganze Hoffnung liegt allein in ihm.\grqq} - Sartre im Interview mit Christian Grisoli: \glqq Entretien avec Jean-Paul Sartre\grqq , Paru 13, Dez. 1945, S. 5-10 \\
  
  Eine der ersten Assoziationen, die einem beim Lesen dieses Zitates in den Sinn kommt, ist das Prinzip des freien Willen des Menschen. Sartre legt seinen Ansichten zugrunde, dass der Mensch einen freien Willen besitze. Aus dieser Annahme heraus folgert er, dass der Mensch die Fähigkeit besitzt, seinen Lebensweg ständig neu zu erfinden. Außerdem ergibt sich daraus die Entscheidungsfähigkeit des Menschen, für sein Handeln vollste Verantwortung zu übernehmen, obwohl man ihn nie fragte, ob er es denn so wirklich wolle\footnote{der Autor kann sich nicht erinnern, dass man ihn zu seiner Geburt oder zu einem späteren Zeitpunkt in seinem Leben einmal fragte, ob er die Fähigkeit, freie Entscheidungen treffen zu können, besitzen wolle}. 
  
  Er bedient sich demnach in seinen Ansichten einem komplett logischem Aufbau, indem er das Prinzip des nicht Vorherbestimmten und die gründsätzliche Entscheidungsfreiheit annimmt. Alles andere ergibt sich daraus. \\
  
  Dementsprechend sei das Ziel die Rettung des Menschen, erreicht durch den selbstbestimmten und damit nicht vorgezeichneten Weg. Dass die Rettung nun auf unterschiedlichste Arten zu bewältigen sei, soll im Folgenden erörtert werden. Außerdem stellt sich die Frage, was unter dem Begriff \textit{Rettung} tatsächlich verstanden werden kann? \\
  
  Angenommen, zwei Menschen sind in ihren Handlungen, Entscheidung und Rettungen in einem bestimmten Definitionsbereich eng miteinander verwoben. Der eine, reicher Aktieninhaber und der andere, Bauarbeiter auf der Baustelle der neuen Luxusvilla des einen. Beide verfolgen sie grundsätzlich unterschiedliche Ziele im Leben, befinden sich somit auf anderen gesellschaftlichen Niveaus. Verallgemeinert und vieles vernachlässigend lassen sich zwei Szenarien beschreiben, wie die beiden miteinander auskommen können. 
  
  Im Ersten arbeitet der Bauarbeiter fleißig, ist bemüht und teilt sein Wissen mit dem in der Baubranche etwas unerfahrenem Aktieninhaber. Im Gegenzug wird er dafür fair entlohnt. Man spricht von einer win-win Situation, wo die Kommunikation zwischen den beiden gut funktioniert und das Resultat dem weiteren Lebensweg der beiden von Nutzen ist. 
  
  Im Zweiten Szenario sieht der Bauarbeiter die Klassenunterschiede der beiden und entwickelt eine Eifersucht\footnote{hervorgerufen durch diverse psychologische Effekte, auf die genauer einzugehen nicht erforderlich ist}, die sich darin erkenntlich zeigt, dass er unsauber und nur das absolut Notwendigste arbeitet. Im Falle, dass dies dem Aktieninhaber auffällt, wird dieser ihn nicht fair entlohnen (wollen) und ihn im schlimmsten Falle des Bauauftrags entziehen. In diesem Fall ist das Resultat für den weiteren Lebensweg der beiden nicht von großem Nutzen, da es beide zum Beispiel in eine Abwärtsspirale bringen könnte\footnote{es wird vernachlässigt, dass dem Aktieninhaber die Kündigung seines Bauarbeiters wenig Sorgen bereitet, was in der Realität jedoch zumeist der Fall wäre - aber es wurde ja auch nicht der genaue Hintergrund des Aktieninhabers beleuchtet, weswegen diese Vernachlässigung als legitim angesehen werden kann}.\\
  
  Die Frage, die sich nun stellt, ist, inwiefern nun eine Rettung erfolgte? Die Rettung ist das Ziel und wenn man in Retroperspektive auf eine bereits geschehene Handlung sieht, wird sie zum Resultat. Da nun offensichtlich zwei Resultate ersichtlich sind, impliziert dies das Vorhandensein mehrerer Rettungen. Sartres theoretischem Konstrukt zufolge ist dies möglich, doch wurde noch immer nicht erörtert, ob dies Sinn mache und was man unter einer Rettung nun genau verstehen könne? \\
  
  Im Laufe der Weltgeschichte haben sich die Menschen unterschiedliche Theorien dazu angeeignet. Der Lehre der katholischen Kirche zufolge erfolgt eine Rettung des Menschen durch Jesus Christus - eine externe Person, an deren Wahrhaftigkeit ein Konstrukt an Annahmen geheftet ist, das in sich geschlossen Sinn macht und deswegen für eine Beantwortung der Rettungsfrage herangezogen werden kann. Neben der Ansichtsweise der katholischen Kirche gibt es natürlich noch viele andere. Die Tatsache, dass all diese Ansichten der Rettung nebeneinander existieren (können) kann als Beweis dafür angesehen werden, dass es viele Rettungen des Menschen gibt. Zu bedenken gilt, dass nur in einer Gesellschaft, in der die Menschenrechte vollständig gewahrt werden, dieses von Sartre angesprochene Prinzip der Rettungen komplett zum Tragen kommt. Denn eine gewichtige Rolle spielt hierbei der freie Wille und damit die Freiheit, zu entscheiden, welche Religion man als Rettungsvorlage heranzieht. Als Beispiel kann hier Österreich gesehen werden, da hier jeder Mensch seine Religion frei wählen kann. 
  
  Die Aufgabe des vorliegenden Essay sollte aber nicht die Behandlung dieser Fragestellung sein, sondern die der Rettungsfrage an sich.\\
  
  Was der Autor zeigen will, ist, dass das Ziel des Weges, das Sartre nennt, nicht vorherbestimmt und keineswegs als absolut zu bezeichnen ist. Es gilt somit zu bedenken, dass es unterschiedliche Rettungen gibt\footnote{die in jedem Moment zu Resultaten werden}, wobei diese auf keinen Fall in ihrer Richtigkeit gleichzusetzen und damit gleichwertig sind\footnote{auf eine Einteilung in gute und böse Rettungen wird verzichtet, da es als nicht zweckdienend für die Erörterung des Zitates erachtet wird, da es einer ausholenderen Erklärung mit einhergehenden Defintionen bedürfe}, wie am Beispiel des Bauarbeiters illustriert. 
  
  Eine Sache fehlt noch im Zitat bzw. kann der Autor sie nicht direkt herauslesen, vorausgesetzt, Sartre habe sie versteckt eingebaut. Der Autor meint den Rückkoppelungseffekt der Rettung zum Weg. Wenn es einen nicht vorgezeichneten Weg zur Rettung gibt und man grundsätzlich frei ist, diesen zu finden oder zu erfinden, impliziert dies, dass man jeden Moment eine Rettung durchlebt und, sofern man die Fähigkeit des Speicherns von Inhalten aus der Vergangenheit mit einhergehendem Abrufen in der Gegenwart besitzt\footnote{= Fähigkeit des Nachdenkens über Vergangenes - Reflexion}, in der Lage ist, die Rettung bzw. das Ziel zu adaptieren und somit konstant neu zu erfinden. \\
  
  Um kurz die bisherigen Erkenntnisse zusammenzufassen: Das Zitat von Sartre basiert auf den Annahmen der Unvorherbestimmtheit und der freien Entscheidungsfähigkeit. Als Ziel des Weges kann die Rettung des Menschen angesehen werden. 
  
  Hinzugefügt wird, dass es unterschiedliche Arten von Rettungen gibt, die nicht gleichwertig sind. Außerdem wurde der Rückkoppelungseffekt, der zwischen Rettung und Weg besteht, eingeführt. Aus diesem geht hervor, dass man jeden Moment eine Rettung durchlebt und diese somit zu einem Resultat wird, das in einem Gedächtnis mehr oder weniger festgehalten wird und für Analysen jeglicher Art zur Verfügung steht. \\
  
  An dieser Stelle wird versucht, eine allgemeine Theorie über die eben beschriebenen Zusammenhänge zu legen. Zu Beginn wird die Theorie beschrieben und anschließend eine Darstellungsform entwickelt, die hilft, die Theorie zu verstehen. Mithilfe dieser Theorie und Darstellungsform soll vor allem eine neue Sichtweise auf das Konzept der Rettungen und Entscheidungen gewonnen werden, die die bereits gestellte Frage, ob eine Rettung grundsätzlich Sinn mache, beantworten sollte. \\
  
  Das Leben kann als eine Minimierungsfunktion angesehen werden. Die Größe, die dabei minimiert werden sollte, ist die Rettung. Das bedeutet, dass jede Entscheidung mit damit einhergehender Handlung darauf abzielt, das Resultat zu einem Minimum zu machen - jede Entscheidung mit gesetzer Handlung führt dementsprechend auch zu einem Minimum! Die Bedeutung ergibt sich in der verglichenen Höhe der einzelnen Minima. 
  
  Nun wird vorgeschlagen, wie dieser Sachverhalt graphisch dargestellt werden kann. Der Autor schlägt ein sogenanntes Rettungs-Diagramm\footnote{die Bezeichnung wurde willkürlich vom Autor festgelegt} vor. Auf der y-Achse werden die Resultate aufgetragen, die x-Achse ist (fast) dimensionslos\footnote{der Autor verzichtet an dieser Stelle auf eine Dimension, wobei bei genauere Betrachtung sie dennoch eine Dimension enthält}.
  
  Grundsätzliches Vorgehen für die Erstellung eines Diagramms: jedes, jeweils ein Minimum repräsentierendes, mögliches Resultat einer Handlung wird in einer unterschiedlichen Höhe aufgetragen. Anschließend werden diese eingetragen Höhenlinien durch Linien miteinander verbunden. Nach abschließender Beschriftung könnte das Diagramm so wie in Abbildung 1 aussehen. \\
  
  Es folgt eine nähere Beschreibung des Diagramms. Die Verbindungslinien zwischen den einzelnen Minima entsprechen den gesetzten Handlungen. Ein Maximum entspricht der größten gesetzten Handlung. Die Steigung wird definiert als Betrag der Änderung der Handlung nach der Änderung der i-ten Rettung:
  
  \[|\frac{dH}{dR_i}| \] 
  
  \begin{itemize}[label=]
    \item H ... Handlung
    \item $R_i$ ... i-te Rettung, die gerade durchlebt wird
  \end{itemize}
  
  Die, um die Steigung interpretieren zu können eingeführte Form der i-ten Rettung entspricht nicht der Rettung bzw. dem Resultat, das angestrebt wird, sondern beschreibt, dass man in jedem Moment eine Rettung durchlebt, die zusammen mit den Entscheidungen alias Handlungen auf ein Minimum (längerfristiges Resultat) abzielt. \\
  
  \ENsetup{
    scale=1.5, 
    axes=y, 
    y-label-text=\footnotesize Resultat
  }
  
  \begin{figure}[!htbp]
    \begin{endiagram}
      \ENcurve{6,4,3,7,5,6}
      \ShowNiveaus[niveau={N1-2, N1-3, N1-4, N1-5}]
      \node[above,xshift=4pt] at (N1-2) {Minimum 3} ;
      \node[below] at (N1-3) {Minimum 1} ;
      \node[above,xshift=4pt] at (N1-4) {Handlung 2} ;
      \node[below] at (N1-5) {Minimum 2} ;
    \end{endiagram}
    \caption{Rettungs-Diagram zur Verdeutlichung}
  \end{figure}
  
  Die Steigung kann grundsätzlich in zwei Arten gedeutet werden. Bewegt man sich von einem Minimum zum anderen und es liegt ein Schritt größter Handlung dazwischen (Maximum), bedeutet die Größe der Steigung die Anstrengung, die notwendig ist, um ein Maximum zu erreichen. Hat man dieses erreicht, bedeutet die Steigung, mit welcher Sicherheit man sich auf ein Minimum zubewegt. Das Rettungs-Diagramm kann somit in zwei Richtungen gelesen werden\footnote{z.B. von Minimum 1 zu Minimum 2 und von Minimum 2 zu Minimum 1}. \\
  
  Ein Minimum besitzt die Eigenschaft, dass die Steigung gleich null ist\footnote{$|\frac{dH}{dR_i}| = 0$}. Es tritt somit keine Änderung des i-ten Resultats mehr ein, was bedeutet, man befindet sich in einem statischen Zustand. Oder, um es anders zu formulieren, man befindet sich in einem Zustand der Zufriedenheit.
  
  Interessant wird es, wenn man sich die Änderung der Handlung ansieht. Kann sie sich dennoch ändern? - Ja! Nur weil der Mensch in einem Zustand der Zufriedenheit verweilt, bedeutet dies nicht, dass er keine Entscheidungen mehr treffen kann\footnote{er trifft nämlich immer welche}, es heißt nur, dass jede getroffene Entscheidung seinen Zustand nicht wesentlich ändert bzw. den Mensch unter Umständen nur darin bestärkt, in diesem Minimum zu verweilen. Natürlich kann auch der Fall eintreten, dass keine Handlungen mehr gesetzt werden.\\
  
  Was passiert nun, wenn man von einer Stelle, an der die Steigung gleich null ist, sich etwas bewegt, also Entscheidungen setzt, die gleichzeitig ein dem Minimum ungleiches Resultat erzeugen? Dann gibt es zwei Fälle. Entweder es bedarf des Absolvierens einer größtmöglichen Handlung, die einem zu einem anderen Minimum bringt oder man bewegt sich automatisch plötzlich auf ein Minimum zu. In diesem zweiten Fall \textit{rastete} man auf einem Plateau, das bedeutet, man musste sich nur minimalst bewegen, um zu einem tatsächlichen Minimum zu kommen\footnote{es wird auf eine mathematische Definition eines Plateaus (auch Sattelstelle oder Terrassenpunkt genannt) verzichtet, da dies in diesem Kontext als nicht interpretierbar und somit nicht zweckbringend erachtet wird}. Ein Beispiel für eine Sattelstelle repräsentiert Minimum 3 in Abbildung 1. \\
  
  Betrachten wir uns die einzelnen Höhen der Minima und stellen dazu die Frage, welches Minimum nun erstrebenswerter sei? - Die Antwort ist einfach: jenes Minimum, das unter allen anderen liegt\footnote{in Abbildung 1 wäre dies Minimum 1, da es unter Minimum 2 liegt}. Es wird zu einem globalen Minimum, das zu erreichen das Ziel des Menschen ist. Es kann in einem bestimmten Definitionsbereich als Zustand größtmöglicher Zufriedenheit gedeutet werden. \\
  
  An dieser Stelle sieht der Autor die Entwicklung der Theorie und die grundlegende Erklärung der Rettungs-Diagramme für beendet. Er möchte die Diagramme in einem Fallbeispiel zur Analyse verwenden, um ihre Aussagekraft zu veranschaulichen und vielleicht einiges verständlicher darzustellen. \\
  
   Das herangezogene Szenario ist dasselbe, wie zu Beginn gezeigt. Die Akteure sind der Bauarbeiter und der Aktieninhaber. Der Bauarbeiter wird im Folgenden mit B, der Aktieninhaber mit A abgekürzt.
   
   In der Beziehung zwischen A und B gibt es grundsätzlich zwei Möglichkeiten. Einerseits können sie gut miteinander auskommen, andererseits können sie sich streiten und nicht gut miteinander auskommen. Tritt erstgenannter Fall ein, befinden sie sich in Situation M1, tritt zweiter Fall ein in M2. Sie können von M1 in M2 gelangen, indem A und B miteinander kommunizieren, wobei dies auf diverse Arten erfolgen kann und demnach nicht auf das direkte Gespräch zu beschränken ist. Auch Informationen, die über Dritte an die jeweilige Partei kommen und eine Entscheidung mit Resultat zur Folge haben, spielen hierbei eine Rolle. 
   
   Auch ist eine Bewegung von M2 nach M1 möglich, ebenfalls durch Kommunikation. Da jedoch die Richtung, in der man sich bewegt unterschiedlich ist, nimmt diese Form der Kommunikation eine andere Gestalt an wie die soeben beschriebene. 
   
   Der Theorie zufolge ist es schwieriger von M1 in M2 zu kommen, wie umgekehrt. Zu beachten ist, dass die Höhen der Minima vom Autor gesetzt wurden und somit seinen Standpunkt reflektieren\footnote{dass, wenn A und B gut miteinander auskommen dies ein erstrebenswerterer Zustand wie umgekehrt ist}. Aus der Sicht von A und B kann das Rettungs-Diagramm natürlich anders aussehen. \\ 
   
  \begin{figure}[!htbp]
    \begin{endiagram}[scale=2]
      \ENcurve{5,2,8,4,5}
      \ShowNiveaus[niveau={N1-2, N1-3, N1-4}]
      \node[below] at (N1-2) {M1 - A und B kommen gut aus} ;
      \node[above,xshift=4pt] at (N1-3) {H1 - Kommunikation zwischen A und B} ;
      \node[below] at (N1-4) {M2 - Streit zwischen A und B} ;
    \end{endiagram}
    \caption{Rettungs-Diagram von Szenario 1}
  \end{figure}
  
  Die eben erwähnte Parallelexistenz von Rettungs-Diagrammen bezogen auf eine Situation kann verwendet werden, um durch Überlagerung einzelner Diagramme ein Minimum zu finden, das repräsentativ ist für die Menschheit an sich und somit unter Umständen auch für das Leben. Eine solche vorgeschlagene Überlagerung soll in Abbildung 3 dargestellt werden.
  
  Um die Überlagerung durchzuführen, werden zunächst die Rettungs-Diagramme aus den Sichtweisen der einzelnen direkt betroffenen Parteien gezeichnet und die Minima beschriftet. In Abbildung 3 wird das Rettungs-Diagramm aus Sicht von A grob gepunktet dargestellt, jenes aus Sicht von B fein gepunktet. Werden beide Diagramme so wie in Abbildung 3 in ein Diagramm gezeichnet, muss eines gespiegelt werden, sodass jene Minima, die in diesem Fall die Positionen der beiden in Bezug auf streiten und nicht streiten repräsentieren, jeweils übereinander positioniert sind. Das Maximum H1 ist die Spiegelachse und bleibt somit an gleicher Stelle. 
  
  \begin{figure}[!htbp]
    \begin{endiagram}[scale=1.5]
      \ENcurve[tikz=dotted]{5,2,8,4,5}
      \ENcurve[tikz=densely dotted]{5,4,8,2,5}
      \ENcurve{5,-2,8,0,5}
      \ShowNiveaus[niveau={N1-2, N1-3, N1-4, N2-2, N2-3, N2-4}]
      \node[below] at (N1-2) {M1} ;
      \node[above,xshift=4pt] at (N1-3) {H1} ;
      \node[below] at (N1-4) {M1*} ;
      \node[below] at (N2-2) {M2*} ;
      \node[above,xshift=4pt] at (N2-3) {H1} ;
      \node[above,xshift=4pt] at (N2-4) {M2} ;
    \end{endiagram}
    \caption{Rettungs-Diagram von Szenario 1 mit Überlagerungen}
  \end{figure}
  
  \begin{itemize}[label=]
    \item M1  ... A und B kommen gut miteinander aus (Sicht: A)
    \item M1* ... A und B streiten miteinander (Sicht: A)
    \item M2  ... A und B streiten miteinander (Sicht: B)
    \item M2* ... A und B kommen gut miteinander aus (Sicht: B)
    \item H1  ... Kommunikation zwischen A und B
  \end{itemize}
  
  Da es sich bei der Parteienanzahl um eine gerade Zahl handelt, muss eine dritte Person eingeführt werden, die das Geschehen von aussen bewertet. Er schlägt sich auf eine Seite und sagt, welche Ansichtsweise ihm besser gefällt. Ihm ist es komplett frei, zu sagen, ob seine Meinung bereits durch ein Diagramm vollständig repräsentiert wird und wenn nicht, dann kann er selbst seines gemäß obigen Regeln einzeichnen. Im vorliegenden Fall ist der externe Beobachter der Autor und schlägt sich auf die Seite des Aktieninhabers (Linie M1 und M1*). 
  
  Um herauszufinden, wie die Sache nun in Bezug auf das Leben generell zu sehen ist, werden die Graphen einfach addiert, so wie in Abbildung 3 gezeigt. Die erhaltene relative Position der Minima gibt an, welches Resultat erstrebenswerter sei. In diesem Fall kommen wir zum Ergebnis, dass ein gutes Klima zwischen A und B erstrebenswert ist! Das Ergebnis ist universell und resistent gegenüber der Vertauschung der Standpunkte der einzelnen Parteien.\\
  
  Um nach etlichen Ausführungen zum Ende zu kommen: Sartre liefert mit seinem Zitat eine Aussage, die sehr allgemein formuliert ist und die Grundannahmen für die Entwicklung der eben beschriebenen Theorie bildet. Dieser Theorie zufolge ist das Leben eine Minimierungsfunktion, was bedeutet, dass jede Handlung das Ziel hat, ein Minimum von Resultaten bzw. Rettungen zu erreichen\footnote{durch das freie Treffen von Entscheidungen und Durchführen von Handlungen}. Es kann somit mit diesen Diagrammen jegliche beliebige Lebenssituation in vielerlei Hinsicht analysiert werden\footnote{Analyse der Zukunft - welche Handlung zu bevorzugen sei, Analyse der Vergangenheit, Analyse des Lebens, Staatsanalyse, ...}. 
  
  Ein globales Minimum jedoch kann aus einer objektiven Perspektive heraus nie erreicht werden, es sind im Moment nur Annäherungen denkbar. Wenn man sich in einem religiösem Rahmen bewegt, existiert unter Umständen sogar ein globales Minimum, das jedoch stark von der Religion und ihren Zielen sowie Inhalten abhängt. \\
  
\end{document}